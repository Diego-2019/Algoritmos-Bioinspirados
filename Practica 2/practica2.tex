\documentclass{report}
\usepackage[spanish]{babel}
\usepackage[margin=2cm]{geometry}
\usepackage{graphicx}
\usepackage{float}
\usepackage{titlesec}
\usepackage{caption}
\usepackage{listings}
\usepackage{xcolor}

\definecolor{codegreen}{rgb}{0,0.6,0}
\definecolor{codegray}{rgb}{0.5,0.5,0.5}
\definecolor{codepurple}{rgb}{0.58,0,0.82}
\definecolor{backcolor}{rgb}{0.95,0.95,0.95}


\lstset{
    basicstyle=\ttfamily,
    inputencoding=utf8,
    extendedchars=true,
    literate=%
    {á}{{\'a}}1
    {é}{{\'e}}1
    {í}{{\'i}}1
    {ó}{{\'o}}1
    {ú}{{\'u}}1
    {ñ}{{\~n}}1
    {Á}{{\'A}}1
    {É}{{\'E}}1
    {Í}{{\'I}}1
    {Ó}{{\'O}}1
    {Ú}{{\'U}}1
    {Ñ}{{\~N}}1
}


\lstdefinestyle{mystyle}{
    backgroundcolor=\color{backcolor},
    commentstyle=\color{codegreen},
    keywordstyle=\color{red},
    numberstyle=\tiny\color{codegray},
    stringstyle=\color{codepurple},
    basicstyle=\ttfamily\footnotesize,
    breakatwhitespace=false,
    breaklines=true,
    captionpos=b,
    keepspaces=true,
    numbers=left,
    showspaces=false,
    showstringspaces=false,
    showtabs=false,
    tabsize=2  
}

\titleformat{\section}
{\huge\bfseries}{\thesection.}{1em}{}
\titleformat{\subsection}
{\large\bfseries}{\thesubsection}{1em}{}

\renewcommand\thesection{\arabic{section}}

\title{\Huge{\textbf{Practica 2. Algoritmo genético para codificación de permutaciones.}}\\
\Large{\textbf{Algoritmos Bioinspirados}}}
\author{Diego Castillo Reyes\\Marthon Leobardo Yañez Martinez\\Aldo Escamilla Resendiz}

\graphicspath{{Imagenes/}}
\begin{document}
    \maketitle
    \tableofcontents
    \newpage

    \section{Introducción}
    En esta práctica se implementó un algoritmo genético para resolver el problema de codificación de 
    permutaciones. En especifico encontrar las combinaciones para las soluciones de un cuadrado mágico.
    El cuadrado mágico se refiere a una matriz cuadrada de números enteros en la que la suma de los números 
    en cada fila, columna y diagonal es la misma.\\

    Por ejemplo: 
    %Ejemplo de un cuadrado mágico
    \begin{table}[H]
        \centering
        \begin{tabular}{|c|c|c|}
            \hline
            8 & 1 & 6\\
            \hline
            3 & 5 & 7\\
            \hline
            4 & 9 & 2\\
            \hline
        \end{tabular}
        \caption{Ejemplo de un cuadrado mágico}
    \end{table}
    En este ejemplo la suma de los números en cada fila, columna y diagonal es 15.\\
    La idea del algoritmo genético es encontrar la permutación de los números del 
    1 al 9 que formen un cuadrado mágico de tamaño nxn.\\

    \section{Desarrollo}
    Para la implementación del algoritmo genético se utilizó el lenguaje de programación Python.
    %Añade codigo p2.py
    \lstinputlisting[language=Python, style=mystyle]{p2.py}
    \section{Conclusión}
    •  ¿Cuál fue la función objetivo que encontró más rápido el cuadrado mágico?
    La función que efectivamente puede encontrar más rápido un cuadrado mágico es la parte 
    de la función verificarexito que busca una aptitud de 0.0
    Esto se debe a que esta condición identifica directamente un cuadrado mágico perfecto sin errores, y la detección es inmediata 
    tan pronto como se encuentre un individuo con esta aptitud en la población. La evaluación se realiza en cada generación, 
    y tan pronto como se encuentra una aptitud de 0.0, el algoritmo puede terminar, lo que es la manera más rápida de concluir 
    la búsqueda bajo las condiciones ideales.

    La búsqueda de un error mínimo no garantiza un cuadrado perfecto y, además, puede continuar evaluando más individuos incluso después de encontrar cuadrados casi perfectos, lo cual es menos eficiente si el objetivo es encontrar cuadrados mágicos absolutamente precisos.

    •  ¿Cómo modificaría el algoritmo para encontrar todos los posibles cuadrados mágicos? (recuerde que existe más de una solución)
    Diseñar operaciones de cruce que preserven características clave de los cuadrados mágicos, como las sumas de filas, columnas y diagonales. Esto puede ayudar a mantener la viabilidad de las soluciones a través de generaciones.
    Ademar de conservar siempre una copia de los mejores individuos encontrados hasta ahora para asegurarte de que las buenas soluciones no se pierdan a lo largo de las generaciones.

    •  ¿Modificó  los  parámetros  de  su  algoritmo  para  los  diferentes  tamaños  del  cuadrado  mágico(n= 4yn= 5)? Si su respuesta es afirmativa ¿cuáles fueron los parámetros?
    Si, se utilizó parametros para n = 3 y 4, ya que el tiempo de ejecucion con otros parametros era bastante e incluso llegaba a romperse el programa.
    

\end{document}